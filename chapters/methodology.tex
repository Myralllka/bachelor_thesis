\chapter{Methodology}

\label{chapter:methodology}

% Camera calibration
% This is done by knowing the 3D position of each interest point and it's projections (colored sircles on \autoref{fig:calib}), and after taking a sample of images on a static camera and the pattern moving along X and Y axis with different scale and skew.
% Then some algorithm can be used to estimate unknown parameters of a calibration matrix, for example: estimate the camera projection matrix (can be done from 6 correspondences) and decompose it to $R$, $\vec{t}$ and $K$ 

% In ROS, there is a package that can be used for a camera calibration\footnote{\href{http://wiki.ros.org/camera_calibration}{ROS camera calibration package}}. As input it takes a chessboard parameters (square size and number of squares) then in an interactive mode it collects images for calibration and as a result save a file with camera calibration matrix and distortion coeficiants (if needed).

\section{General multicamera calibration}

\section{Features extraction and matching}

\section{Features pose estimation}

